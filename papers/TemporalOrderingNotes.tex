\documentclass[11pt,letterpaper]{article}
\usepackage{fullpage}
\usepackage[pdftex]{graphicx}
\usepackage{amsfonts,eucal,amsbsy,amsopn,amsmath}
\usepackage{url}
\usepackage[sort&compress]{natbib}
\usepackage{latexsym}
\usepackage{wasysym} 
\usepackage{rotating}
\usepackage{fancyhdr}
\DeclareMathOperator*{\argmax}{argmax}
\DeclareMathOperator*{\argmin}{argmin}
\usepackage{sectsty}
\usepackage[dvipsnames,usenames]{color}
\usepackage{multicol}
\definecolor{orange}{rgb}{1,0.5,0}
\usepackage{multirow}
\usepackage{tabularx}
\usepackage[table]{xcolor}
\usepackage{sidecap}
\usepackage{caption}
\usepackage{times}
\usepackage{amsmath}
\usepackage{floatrow}
\usepackage{algorithm}
\usepackage{algorithmic}
\usepackage{appendix}
\usepackage{tikz}
\usepackage{rotating}
\usepackage{array}

\usetikzlibrary{arrows}

\newcolumntype{+}{>{\global\let\currentrowstyle\relax}}
\newcolumntype{^}{>{\currentrowstyle}}
\newcommand{\rowstyle}[1]{\gdef\currentrowstyle{#1}%
#1\ignorespaces
}

\renewcommand{\captionfont}{\small}
\setlength{\oddsidemargin}{-0.04cm}
\setlength{\textwidth}{16.59cm}
\setlength{\topmargin}{-0.04cm}
\setlength{\headheight}{0in}
\setlength{\headsep}{0in}
\setlength{\textheight}{22.94cm}
\allsectionsfont{\normalsize}
\newcommand{\ignore}[1]{}
\newenvironment{enumeratesquish}{\begin{list}{\addtocounter{enumi}{1}\arabic{enumi}.}{\setlength{\itemsep}{-0.25em}\setlength{\leftmargin}{1em}\addtolength{\leftmargin}{\labelsep}}}{\end{list}}
\newenvironment{itemizesquish}{\begin{list}{\setcounter{enumi}{0}\labelitemi}{\setlength{\itemsep}{-0.25em}\setlength{\labelwidth}{0.5em}\setlength{\leftmargin}{\labelwidth}\addtolength{\leftmargin}{\labelsep}}}{\end{list}}

\bibpunct{(}{)}{;}{a}{,}{,}
\newcommand{\nascomment}[1]{\textcolor{blue}{\textsc{\textbf{[#1 --nas]}}}}
\newcommand{\wvmcomment}[1]{\textcolor{green}{\textsc{\textbf{[#1 --wvm]}}}}

\newcommand{\two}[2]{\begin{tabular}{c}\textbf{#1}\\\textbf{(#2)}\end{tabular}}

\newcommand{\black}{\cellcolor{black}}

\setlength{\tabcolsep}{5pt}

\pagestyle{fancy}
\lhead{}
\chead{}
\rhead{}
\lfoot{}
\cfoot{\thepage}
\rfoot{}
\renewcommand{\headrulewidth}{0pt}
\renewcommand{\footrulewidth}{0pt}

\DeclareFloatFont{small}{\small}
\floatsetup[table]{font=small}
\floatsetup[caption]{font=small}
\floatsetup[algorithm]{font=small}

\title{Temporal Ordering Notes}

\date{Spring 2014}

\begin{document}
\maketitle

\section{Structured SVM}

We have a generic structured SVM implemented according to the following objective
function which minimizes the hinge loss over a set of $N$ structured inputs and
outputs $\{(\mathbf{x}_i,\mathbf{y}_i)|1\leq i \leq N\}$:

\begin{equation}
\begin{split}
\label{objective}
& \min_{\mathbf{w},  \mathbf{b}} \lambda_2\|\mathbf{w}\|_2^2+\lambda_1\|\mathbf{w}\|_1 \\
 & +\sum_{i=1}^N\bigg(-\mathbf{w}^\top \mathbf{g}(\mathbf{x}_i,\mathbf{y}_i)-\mathbf{b}^\top \mathbf{l}(\mathbf{y}_i)+\max_{\mathbf{y}\in \mathcal{Y}_{x_i}}\Big(\mathbf{w}^\top\mathbf{g}(\mathbf{x}_i,\mathbf{y})+\mathbf{b}^\top \mathbf{l}(\mathbf{y})+c(\mathbf{x}_i,\mathbf{y}_i,\mathbf{y})\Big)\bigg)
\end{split}
\end{equation}

The components of $\mathbf{w}$ are the regularized weights on features 
$\mathbf{g}$, the components of $\mathbf{b}$ are biases on label counts 
$\mathbf{l}$ and $c(\mathbf{x}_i,\mathbf{y}_i,\mathbf{y})$ is the cost of predicting
$\mathbf{y}$ for example $i$.  The minimization of objective \ref{objective} 
results in feature weights $\mathbf{w}$ and biases $\mathbf{b}$ which
make a prediction $\mathbf{y}$ for input $\mathbf{x}$ according to:

\begin{equation}
\label{prediction}
\mathbf{y}=\argmax_{\mathbf{y}'\in\mathcal{Y}_x}\Big( \mathbf{w}^\top \mathbf{g}(\mathbf{x},\mathbf{y}')+\mathbf{b}^\top\mathbf{l}(\mathbf{y}') \Big)
\end{equation}

The model assumes that each structured input $\mathbf{x}_i$ factors into components $(x_{i1},\hdots,x_{ij},\hdots,x_{ik(\mathbf{x}_i)})$, 
and each structured output $\mathbf{y}_i$ factors into corresponding labels
 $(y_{i1},\hdots,y_{ij},\hdots,y_{ik(\mathbf{x}_i)})$.  Each label $y_{ij}$ 
 comes from a set of factor labels $\mathcal{L}$ (and so $\mathcal{Y}_x=\mathcal{L}^{k(x_i)}$).
Furthermore, the model is constructed so that $\mathbf{g}$ and 
$\mathbf{w}$ have components $g_{ml}$ and $w_{ml}$ where $m$ indexes a 
feature type and $l\in\mathcal{L}$ is a factor label, with $g_{ml}$ expressible
as:

\begin{equation}
g_{ml}(\mathbf{x}_i,\mathbf{y}_i)=\sum_{j=1}^{k(\mathbf{x}_i)}\mathbf{1}(y_{ij}=l)f_m(x_{ij})
\end{equation}

Where each $f_m$ is a percept that only depends on a single factor of the 
input structure (Nathan's terminology).

Similarly, the model assumes that the cost function $\mathbf{c}$ is factorable as:

\begin{equation}
c(\mathbf{x}_i,\mathbf{y}_i,\mathbf{y}_h)=\sum_{j=1}^{k(\mathbf{x}_i)}\mathbf{1}(y_{ij}\neq y_{hj})
\end{equation}

When computing the loss for each example, the model must search a potentially large $\mathcal{Y}_x$ to find the value of the following expression:

\begin{equation}
\max_{\mathbf{y}\in \mathcal{Y}_{x_i}}\Big(\mathbf{w}^\top\mathbf{g}(\mathbf{x}_i,\mathbf{y})+\mathbf{b}^\top \mathbf{l}(\mathbf{y})+c(\mathbf{x}_i,\mathbf{y}_i,\mathbf{y})\Big)\bigg)
\end{equation}

To perform this search, the model relies on the fact that this expression is equal to:

\begin{equation}
\label{exp}
\max_{\mathbf{y}_h\in \mathcal{Y}_{x_i}}\Bigg(\sum_{j=1}^{k(x_i)}\sum_{l\in\mathcal{L}}\bigg( \mathbf{1}(l=y_{hj})\Big( \mathbf{1}(y_{ij}\neq y_{hj})+\sum_m w_{ml}f_m(x_{ij})\Big) \bigg)\Bigg)
\end{equation}

In order to compute the value of expression \ref{exp}, the model computes the following
for each $i$, $j$, and $l$:

\begin{equation}
u_{ijl} = \mathbf{1}(y_{ij}\neq l)+\sum_m w_{ml}f_m(x_{ij})
\end{equation}

Notice that these values are used in expression \ref{exp}.  They are passed to a domain-specific optimization function that computes $\mathbf{y}_h$ in expression \ref{exp} through:

\begin{equation}
\label{ilp}
\mathbf{e}_i=\argmax_{\mathbf{e}_i'\in\{0,1\}^{k(x_i)}}\mathbf{e}_i'^\top \mathbf{u}
\end{equation}

Where each component $e_{ijl}$ of $\mathbf{e}_i$ indicates whether 
$y_{hj}=l$, and these components are constrained so that the resulting 
$\mathbf{y}_h\in\mathcal{Y}_{x_i}$.  The resulting $\mathbf{y}_h$ can be
used to determine the value of expression \ref{exp}.

The prediction for an example according to to equation \ref{prediction} is computed 
similarly, just without the cost term in $u_{ijl}$

The implementation allows the model to remain abstract while referring to 
domain specific features and domain specific definitions for $\mathcal{Y}_x$.

\subsection{Temporal Relation Classification Structured Model}

The temporal relation classification task can use the model defined above 
with an implementation of equation \ref{ilp} which has an ILP to enforce several 
constraints.  The temporal relations can be partitioned into separate graph 
structures in several ways for this task, but we currently have them 
partitioned by sentence and sentence pairs.  So, we have a set of 
within-sentence temporal relation graphs $\mathcal{X}_s$ and a set of 
between-sentence temporal relation graphs $\mathcal{X}_b$.  Each 
$\mathbf{x}_s\in\mathcal{X}_s$ contains relations between the document 
creation time and all events and all times in a single sentences.  
Each $\mathbf{x}_b\in\mathcal{X}_b$ contains relations between all events 
and all times in two consecutive sentences. $\mathcal{X}_s$ is passed to 
a structured within-sentence model $\mathcal{M}_s$, and $\mathcal{X}_b$ is 
passed to a structured between-sentence model $\mathcal{M}_b$.  The temporal 
relation classifications from $\mathcal{M}_s$ are used to constrain the 
output space of $\mathcal{M}_b$.

For this task, each $e_{ijl}$ from equation \ref{ilp} indicates whether 
event-event or event-time pair $j$ has temporal relation $l$ in graph $i$.  
There are several constraints on the value of each $e_{ijl}$ enforced through 
an ILP that is run for each graph.  To make the constraints easier to 
understand, we'll rewrite each $e_{ijl}$ as $t_{jkl}$ to indicate that there 
is a temporal relation of type $l$ between vertices $j$ and $k$ (implicitly 
in the same example graph $i$).  We have the following constraints:

\begin{enumerate}

\item \textbf{Label On-Off}: $\forall j\neq k, l: t_{jkl}\in \{0,1\}$

\item \textbf{Single Label}: $\forall j\neq k: \sum_{l\in \mathcal{L}}t_{jkl}=1$

\item \textbf{Converse}: $\forall j<k,l: t_{jkl}-t_{kjl'}\leq 0$ where $l'$ is 
the converse of $l$

\item \textbf{Grounded Time-Time}: $t_{jkl}\geq 1$ if $j$ and $k$ index time 
expression vertices that are given relation $l$ according to their grounded 
intervals

\item \textbf{Transitivity}: $\forall \{j,k,m\}: t_{jkl}+t_{kml'}-t_{jml''}\leq 1$ for Allen's 
interval algebra composition relations that 
say $l(j,k)\wedge l'(k,m)\rightarrow l''(j,m)$

\item \textbf{Disjunctive Transitivity}: $\forall \{j,k,m\}: t_{jkl}+t_{kml'}-\sum_{l''}t_{jml''}\leq 1$ 
for Allen's interval algebra composition relations that say 
$l(j,k)\wedge l'(k,m)\rightarrow \bigvee_{l''}l''(j,m)$

\end{enumerate}

In total there are $n(n-1)|\mathcal{L}|$ variables and $O(n^3)$ 
constraints in the ILP for a graph with $n$ vertices.  The current 
implementation has variables $t_{jkl}$ and $t_{kjl}$ representing forward and 
backward links between two vertices.  It might be possible to eliminate the 
backward link variables along with the \textbf{Converse} constraint for 
improved efficiency, but we haven't had time to think through that yet. 

The Allen's interval relations enforced by the \textbf{Transitivity} and 
\textbf{Disjunctive Transitivity} constraints are given at 
\url{http://www.ics.uci.edu/~alspaugh/cls/shr/allen.html} and in the other 
document we passed around through email.  Our implementation has the ability 
to easily turn off and on each of these constraints to see how they 
contribute to the overall performance.

\end{document}
